\documentclass[a4paper,10pt]{report}

\title{Getting Started Guide \\ Kinematics \& Dynamics Library}
\author{Ruben Smits, Herman Bruyninckx}

\usepackage{listings}
\usepackage{amsmath}
\usepackage{url}
\usepackage{todo}
\usepackage{hyperref}

\begin{document}

\maketitle

\chapter{Introduction to KDL}
\label{cha:introduction-kdl}

\section{What is KDL?}
\label{sec:what-kdl}
Kinematics \& Dynamics Library \todo{Explain why KDL has a reason to exist}


\section{Getting support - the Orocos community}
\label{sec:gett-supp-oroc}

\begin{itemize}
\item This document!
\item The website : \url{http://www.orocos.org/kdl}.
\item The mailinglist. \todo{add link to mailinglist}
\end{itemize}

\chapter{Tutorial}
\label{cha:tutorial}

First off all you need to succesfully install the KDL-library. This is
explained in the \textbf{Installation Manual}, it can be found on
\url{http://www.orocos.org/kdl/Installation_Manual}


\section{Geometric Primitives}
\label{sec:geometric-primitives}

\section{Kinematic Structures}
\label{sec:kinematic-structures}

\section{Kinematic Solvers}
\label{sec:kinematic-solvers}

\section{Motion Specification Primitives}
\label{sec:moti-spec-prim}


\todos

\end{document}

%%% Local Variables: 
%%% mode: latex
%%% TeX-master: t
%%% End: 
